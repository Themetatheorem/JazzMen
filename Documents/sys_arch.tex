\documentclass[12pt, titlepage]{article}

\usepackage{xcolor} % for different colour comments
\usepackage{tabto}
\usepackage{mdframed}
\mdfsetup{nobreak=true}
\usepackage{xkeyval}
\usepackage{tabularx}
\usepackage{booktabs}
\usepackage{hyperref}
\hypersetup{
    colorlinks,
    citecolor=black,
    filecolor=black,
    linkcolor=black,
    urlcolor=blue
}
\usepackage[skip=2pt, labelfont=bf]{caption}
\usepackage{titlesec}
\usepackage{float}
\usepackage{caption}
\usepackage{graphicx}
\graphicspath{ {Images/} }
\usepackage{enumitem}


\setcounter{secnumdepth}{4}

\titleformat{\paragraph}
{\normalfont\normalsize\bfseries}{\theparagraph}{1em}{}
\titlespacing*{\paragraph}
{0pt}{3.25ex plus 1ex minus .2ex}{1.5ex plus .2ex}


%% Comments
\newif\ifcomments\commentstrue

\ifcomments
\newcommand{\authornote}[3]{\textcolor{#1}{[#3 ---#2]}}
\newcommand{\todo}[1]{\textcolor{red}{[TODO: #1]}}
\else
\newcommand{\authornote}[3]{}
\newcommand{\todo}[1]{}
\fi

\newcommand{\wss}[1]{\authornote{magenta}{SS}{#1}}
\newcommand{\ds}[1]{\authornote{blue}{DS}{#1}}




\newcommand{\getCurrentSectionNumber}{%
  \ifnum\c@section=0 %
  \thechapter
  \else
  \ifnum\c@subsection=0 %
  \thesection
  \else
  \ifnum\c@subsubsection=0 %
  \thesubsection
  \else
  \thesubsubsection
  \fi
  \fi
  \fi
}


\makeatother
\begin{document}

\title{System Architecture \\ Sentiments Analysis with Twitter \\ Team Jazz Men}
\author{Anagh Goswami 1217426 \\ Meet Pandya 1214306 \\ Jasman Gill  1211554 \\ Jesse Truong  1222722 \\ Jia Xu  1213268 \\}
\date{\today}
	
\maketitle

\pagenumbering{roman}
\tableofcontents
\listoftables
\listoffigures


\begin{table}[htb]
\caption*{\bf Revision History}
\begin{tabularx}{\textwidth}{p{3.5cm}p{2cm}p{3.5cm}X}
\toprule {\bf Date} & {\bf Version} & {\bf Notes} & {\bf Editor}\\
\midrule
January 8, 2017 & 1.0 & Created First Draft & Meet P\\
\bottomrule
\end{tabularx}
\end{table}

\newpage

\pagenumbering{arabic}

\section{Introduction}
\subsection{Overview}
This document is provides a detailed description of System Architecture for this project. The document describes Anticipated Changes as the project progresses which is split up into Likely and Unlikely Changes. Next, the document describes the System Architecture using.......FILL IN HERE

\section{Design Principle}


\section{Anticipated Changes}
\subsection{Likely Changes}
The following changes are likely to occur as the project proceeds:
\begin{enumerate}[label=AC\arabic*]
  \item Currently, the Sentiments Analysis is static, meaning the keyword to be searched for on Twitter is Hard-Coded. Eventually, it will become dynamic, so users will be able to input keyword and thats what the analysis is done on.
  \item The product only supports Sentiments Analysis on a single keyword. Eventually, the product will support more than one keyword in order to do Comparative Analysis.
  \item The web UI currently is hosted locally on tester's computer. Eventually, the web UI will be hosted on a live server.
  \item Based on user feedback, some features will be modified if not, removed. 
\end{enumerate}

\subsection{Unlikely Changes}
The following changes are unlikely to occur:
\begin{enumerate}[label=UC\arabic*]
  \item It is unlikely that the Sentiments Analysis will be done with any API other than the Alchemy Language API
  \item The way the Front End UI is developed is unlikely to change.
\end{enumerate}

\section{Flow Diagram of Web UI}
\begin{figure}[!htb]
\centering
\includegraphics[width=80mm]{flow}
\caption{Flow Diagram}
\label{fig:Flow}
\end{figure}

\section{Decomposition into Components}
\subsection{Home Page}
Main starting page of the Web UI. User will start off from this page. Page provides an overview of what a Sentiments Analysis is about and how it works.

\subsection {Search Bar for Keyword}
A search bar is provided on the main Home page. This is the central feature to be used by an user to search a keyword in and get the sentiment score.

\subsection {Multiple Search Bars}
A toggle feature allows user to switch to two search bars. This is in order to do comparative analysis when user wants to see differences between two interests.



\end{document}