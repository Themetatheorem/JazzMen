\documentclass[12pt]{article}
\usepackage[margin=1in]{geometry}
\usepackage{booktabs}
\usepackage{enumerate}
\usepackage{listings}
\usepackage{titling}
\usepackage{tabularx}
\usepackage[hidelinks=true]{hyperref}
\usepackage{float}
\usepackage{makecell}
% Comments ------------------------------------------------ 
\usepackage{xcolor}
\newif\ifcomments\commentsfalse

\ifcomments \newcommand{\authornote}[3]{\textcolor{#1}{[#3 ---#2]}}
\newcommand{\todo}[1]{\textcolor{red}{[TODO: #1]}} \else
\newcommand{\authornote}[3]{} \newcommand{\todo}[1]{} \fi

\newcommand{\wss}[1]{\authornote{magenta}{SS}{#1}}
\newcommand{\ds}[1]{\authornote{blue}{DS}{#1}} % End Comments
%---------------------------------------------
% Requirement template -------------------------------------------------------
\newcommand{\requirement}[8]{%
\fbox{\parbox{\textwidth}{%
\parbox[t]{.333\textwidth}{\raggedright% 
\textbf{Req. \#}: \refstepcounter{ReqNumCounter} \arabic{ReqNumCounter} \label{#1}}%
\parbox[t]{.333\textwidth}{\centering% 
\textbf{Req. Type}: \ref{#2}}%
\parbox[t]{.333\textwidth}{\raggedleft%
\textbf{Use Case \#}: \ref{#3}}
\newline\\
\textbf{Description}: #4\\\\
\textbf{Rationale}: #5\\\\
\textbf{Fit Criterion}: #6\\\\
\textbf{Priority}: #7 \hfill \textbf{History}: #8
}}}


% ============================ BEGIN DOCUMENT =============================== %
\begin{document}

% Title Page -----------------------------------------------------------------
\title{Problem Statement\\  Sentiments Analysis with Twitter \\Team Jazz Men}
\author{Anagh Goswami 1217426 \\ Meet Pandya 1214306 \\ Jasman Gill  1211554 \\ Jesse Truong  1222722 \\ Jia Xu  1213268 \\}
\date{\today}
\maketitle
\pagenumbering{gobble}
\newpage


\begin{table}[h]
\centering
\caption{Revision History}
\begin{tabular}{|l|l|l|}
\Xhline{2\arrayrulewidth}
\bf Description of Changes & \bf Author & \bf Date\\\hline
Created first version of the  problem statement. & Jasman Gill & September 28, 2016\\\midrule
Edited the information for the new idea & Meet Pandya & December 15, 2016\\\midrule
Edited the document. &Jesse Truong &January 8, 2017\\\bottomrule
Edited the document for final revision &Meet Pandya & April 7, 2017\\\bottomrule
\Xhline{2\arrayrulewidth}
\end{tabular}
\end{table}

\subsection*{Background}
Social media is a place for small talk, discussion and sharing of ideas. Social media includes many different platforms used by hundreds of millions of people across the globe. Big organizations can utilize these platforms in order to imporove their own reputations as well as figuring out different ways to increase their customer base. Using social media, these companies can discover what the public eye thinks of them.\\

\subsection*{Description of Problem}
One of the biggest issue with social media such as Twitter is that there is lot of views and ideas shared that are taken seriously by people who are active on social media and there is no analytical tool to look at majority of these views. People go on social media to look at opinions on subject of their interests, whether it be sports, movies, restaurants, academic institutions or political figures. People read, talk and give their opinions about different topics. Sometimes if you take a look at a whole bunch of opinions all together then there is a higher possibility of that being true with regards to that particular topic. Of course there is some bias involved in social media opinion but due to the sheer size of number of people and their opinions, the bias is overshadowed. \\
Social Media is a major tool people utilize to see what is being said about something they are interested. This is the reason why it is important to come up with an Analytical Tool in order to give people a way to look at overall opinion of a larger consensus on a certain topic. 


\subsection*{Motivation}
Social media comes with endless possibilies to help improve the quality of life for everyone. Social media is apart of everyones day to day lives. Whether you are deciding on which restaurant to eat at, sharing a picture of your pets or bragging about how great your university life is going, social media is there. Organizations use social media now as a cheaper alternative way to focus in order to develop an image of their company and better themselves. The problem is that social media can be limited at times and can be biased  when finding opinions of a topic. In order to have a better understanding of the publics opinion on a topic, there needs to be a bigger consensus. This is where the major motivation for the project comes from; finding a way to complete an analysis using social media platforms to ease the process of finding the opinion.\\


\subsection*{Stakeholders}
The major stakeholders for this project include members of Team Jazzmen, Dr. Wenbo He, and business analysts. This project is designed to help business analysts and the common users to develop opinions towards different topics. Team Jazzmen are stakeholders because we are the ones developing and maintaining the project. Dr. Wenbo is the project supervisor for this project.\\

\subsection*{Context \& Environment}
The project is designed to help businesses and the common users to develop opinions on various topics through social media. The aim is to help them gauge what the public's opinion on current matters is and possibly determine how to better it. \\

\subsection*{Challenges}
The challenges for this project would be figuring out the way to go through social media and getting all the necessary information needed based on a topic or a keyword. We have to work with the information that is freely available on the internet. One of the major issue with social media is all the information can be biased, so the information we collect has to be presented in a manner that it does not come across as factual. Also, picking what kind of organizations to focus on is a challenge because not all organizations are actively discussed on social media platforms.\\

\subsection*{Constraints}
The constraints we face include the amount of industry specific data that is available to us. We have to work with what we’re given and from there we can try to make trends and analyse data. 


 
\end{document}